% !TEX root = C:\Users\Diegodm\Projects\Testing_Procedural_Generation\informe\Memoira_TFG_Latex\Memoria_TFG_Latex\memtfg.tex

\documentclass[spanish,a4paper,12pt,oneside]{extreport}

\usepackage[dvips]{graphicx}
\usepackage[dvips]{epsfig}
\usepackage[utf8]{inputenc}
\usepackage[spanish]{babel}
\usepackage{alltt}

\usepackage[ruled,vlined,commentsnumbered,linesnumbered,inoutnumbered,titlenotnumbered,noend]{algorithm2e}
\SetKwRepeat{Do}{do}{while}

\usepackage{multirow}
\usepackage{array} 
\usepackage{amsfonts}
\usepackage{amsmath}
\usepackage{bigstrut}
\usepackage{booktabs}
\usepackage{caption}
\usepackage{chngpage}
\usepackage{float}
\usepackage{enumitem,lipsum}
\usepackage{graphicx}
\usepackage{lscape}
\usepackage{microtype}
\usepackage{natbib}
\usepackage{pdflscape}
\usepackage{rotating}
\usepackage{subcaption}
\usepackage{ctable}
\usepackage{hyperref}
\usepackage{enumerate}
\usepackage{gensymb}
\usepackage{eurosym}
\usepackage{xcolor}
\usepackage{tabu}
\usepackage{lineno}
%\\linenumbers
%\setlength\linenumbersep{5pt}
%\renewcommand\linenumberfont{\normalfont\tiny\sffamily\color{gray}}

\usepackage[top=2cm, bottom=2cm, left=2cm, right=2cm]{geometry}

\newenvironment{sourcecode}
{\begin{list}{}{\setlength{\leftmargin}{1em}}\item\scriptsize\bfseries}
{\end{list}}

\newenvironment{littlesourcecode}
{\begin{list}{}{\setlength{\leftmargin}{1em}}\item\tiny\bfseries}
{\end{list}}

\newenvironment{summary}
{\par\noindent\begin{center}\textbf{Abstract}\end{center}\begin{itshape}\par\noindent}
{\end{itshape}}

\newenvironment{keywords}
{\begin{list}{}{\setlength{\leftmargin}{1em}}\item[\hskip\labelsep \bfseries Keywords:]}
{\end{list}}

\newenvironment{palabrasClave}
{\begin{list}{}{\setlength{\leftmargin}{1em}}\item[\hskip\labelsep \bfseries Palabras clave:]}
{\end{list}}

\usepackage{bera}% optional: just to have a nice mono-spaced font
\usepackage{listings}
\usepackage{xcolor}

\colorlet{punct}{red!60!black}
\definecolor{background}{HTML}{EEEEEE}
\definecolor{delim}{RGB}{20,105,176}
\colorlet{numb}{magenta!60!black}

\lstdefinelanguage{json}{
    basicstyle=\normalfont\ttfamily,
    numbers=left,
    numberstyle=\scriptsize,
    stepnumber=1,
    numbersep=8pt,
    showstringspaces=false,
    breaklines=true,
    frame=lines,
    backgroundcolor=\color{background},
    literate=
     *{0}{{{\color{numb}0}}}{1}
      {1}{{{\color{numb}1}}}{1}
      {2}{{{\color{numb}2}}}{1}
      {3}{{{\color{numb}3}}}{1}
      {4}{{{\color{numb}4}}}{1}
      {5}{{{\color{numb}5}}}{1}
      {6}{{{\color{numb}6}}}{1}
      {7}{{{\color{numb}7}}}{1}
      {8}{{{\color{numb}8}}}{1}
      {9}{{{\color{numb}9}}}{1}
      {:}{{{\color{punct}{:}}}}{1}
      {,}{{{\color{punct}{,}}}}{1}
      {\{}{{{\color{delim}{\{}}}}{1}
      {\}}{{{\color{delim}{\}}}}}{1}
      {[}{{{\color{delim}{[}}}}{1}
      {]}{{{\color{delim}{]}}}}{1},
}

\begin{document}

\renewcommand\listtablename{Índice de Tablas}    
\renewcommand\listfigurename{Índice de Figuras}    

%%%%%%%%%%%%%%%%%%%%%%%%%%%%%%%%%%%%%%%%%%%%%%%%%%%%%%%%%%%%%%%%%%%%%%%%%%%%%%%
% Primera página
%%%%%%%%%%%%%%%%%%%%%%%%%%%%%%%%%%%%%%%%%%%%%%%%%%%%%%%%%%%%%%%%%%%%%%%%%%%%%%%
\pagestyle{empty}
\thispagestyle{empty}


\newcommand{\HRule}{\rule{\linewidth}{1mm}}
\setlength{\parindent}{0mm}
\setlength{\parskip}{0mm}

\vspace*{\stretch{0.5}}

\begin{center}
\includegraphics[scale=0.8]{images/escuela-ingenieria-tecnologia-original}\\[10mm]
{\Huge Trabajo de Fin de Grado}
\end{center}

\HRule
\begin{flushright}
        {\Huge Generación Procedural de Contenido 2D en Videojuegos} \\[2.5mm]
        {\Large \textit{Procedural Content Generation in 2D videogames}} \\[5mm]
        {\Large Diego Díaz Morón} \\[5mm]


\end{flushright}
\HRule
\vspace*{\stretch{2}}
\begin{center}
  \Large La Laguna, \today
\end{center}

\setlength{\parindent}{5mm}

%%%%%%%%%%%%%%%%%%%%%%%%%%%%%%%%%%%%%%%%%%%%%%%%%%%%%%%%%%
% Profesores
%%%%%%%%%%%%%%%%%%%%%%%%%%%%%%%%%%%%%%%%%%%%%%%%%%%%%%%%%%
\newpage
\thispagestyle{empty}

{\bf FALTA}
D. {\bf Jesus Miguel Torres Jorge}, profesor Titular de Universidad adscrito al Departamento de Nombre del Departamento de la Universidad de La Laguna, como tutor

\bigskip
D. {\bf Manuel Alejandro Bacallado Lopez}, profesor Titular de Universidad adscrito al Departamento de Nombre del Departamento de la Universidad de La Laguna, como cotutor\pagestyle{empty}

\bigskip
\bigskip
{\bf C E R T I F I C A (N)}

\bigskip
\bigskip
Que la presente memoria titulada:

\bigskip
''{\it Generación Procedural de Contenido 2D en Videojuegos}''

\bigskip
\bigskip
\bigskip

\noindent ha sido realizada bajo su dirección por D. {\bf Diego Díaz Morón}.

\bigskip
\bigskip

Y para que así conste, en cumplimiento de la legislación vigente y a los efectos
oportunos firman la presente en La Laguna a \today

%%%%%%%%%%%%%%%%%%%%%%%%%%%%%%%%%%%%%%%%%%%%%%%%%%%%%%%%%%
\newpage
\thispagestyle{empty}

{ \flushright

\begin{LARGE}
Agradecimientos
\end{LARGE}

\hspace{3mm}

\begin{large}
A Manu, Carlos, Jimena, Rothio y el resto de divulgadoras que me dejaron entrar a un mundo fantástico.

A todas las compañeras de clase con las que compartí alegrías y frustraciones.

A mi familia y colegas cercanas por seguir abrazándome. Son lo más importante que tengo.
\end{large}

}
%%%%%%%%%%%%%%%%%%%%%%%%%%%%%%%%%%%%%%%%%%%%%%%%%%%%%%%%
% Licencia
%%%%%%%%%%%%%%%%%%%%%%%%%%%%%%%%%%%%%%%%%%%%%%%%%%%%%%%%

\newpage
\thispagestyle{empty}

\bigskip
\begin{LARGE}
Licencia
\end{LARGE}

\bigskip
\begin{center}
\includegraphics[scale=1.8]{images/by-nc_88x31}\\[5mm]
\end{center}

%%%%%%%%%%%%%%%%%%%%%%%%%%%%%%%%%%%%%%%%%%%%%%%%%%%%%%%%
% Resumen
%%%%%%%%%%%%%%%%%%%%%%%%%%%%%%%%%%%%%%%%%%%%%%%%%%%%%%%%
\newpage 
\thispagestyle{empty}

\begin{abstract}
{\em
El objetivo principal de este trabajo ha sido investigar e implementar algoritmos útiles para la generación de contenido 2D en videojuegos. Para ello hemos creado un prototipo en la herramiento Godot que utiliza métodos de generación de cotenido procedural (PCG) para crear el mapa del propio juego. 

Las técnicas que hemos aportado para la creación del mapa son el diagrama de Voronoi y Random Walk. Juntando ambas se ha creado un mapa del tipo "mazmorra" en el que el jugador se puede mover. Además, se han tomado los tiempos de ejecución de diferentes métodos para la creación del mismo resultado para contrastar las diferencias.

Por último, cabe destacar el uso de técnicas propias del desarrollo ágil. Cumpliendo con los principios SOLID, implementando patrones de diseño y sistemas como Entity-Component System (ECS).
}

\begin{palabrasClave}
Generación de Contenido Procedural, Algoritmo , Godot, Diagrama de Voronoi, Random Walk, ECS, Mazmorra.
\end{palabrasClave}

\end{abstract}
%%%%%%%%%%%%%%%%%%%%%%%%%%%%%%%%%%%%%%%%%%%%%%%%%%%%%%%%%
% Resumen en inglés
%%%%%%%%%%%%%%%%%%%%%%%%%%%%%%%%%%%%%%%%%%%%%%%%%%%%%%%%%
\newpage 
\vspace*{200px}
\thispagestyle{empty}

\begin{summary}
{
  The main objective if this work has been to research and implement usefull algorithms for generate 2D content in cideogames. For this reason we have created a prototype on Godot. It use Procedural Content Generation (PCG) methods for creating the videogame map. 

  The tecniques that we have used to map creation are Voronoi's diagram and Random Walk. By jointing both algorithms a "dungeon" type of map we have created where the player can move around. Moreover, we have timed the execution times of different creation methods in order to contrast the differences. 

  Finally, agile development tecniques should be highlighted. Fullfilling with SOLID principles, implementing design patterns and systems as Entity-Component System (ECS).
}

\em
\begin {keywords}
Procedural Content Generation, Algorithm , Godot, Voronoi Diagram, Random Walk, ECS, Dungeon.
\end {keywords}

\end{summary}
%%%%%%%%%%%%%%%%%%%%%%%%%%%%%%%%%%%%%%%%%%%%%%%%%%%%%%%%%
\newpage{\pagestyle{empty}}
\thispagestyle{empty}

%%%%%%%%%%%%%%%%%%%%%%%%%%%%%%%%%%%%%%%%%%%%%%%%%%%%%%%%%
\pagestyle{myheadings} %my head defined by markboth or markright
% No funciona bien \markboth sin "twoside" en \documentclass, pero al
% ponerlo se dan un montón de errores de underfull \vbox, con lo que no se
% ha puesto.


%%Aqui debería poner el nombre del proyecto pero, como es muy grande no cabe y se ve feo en el PDF
\markboth{xxxxx}{}

%%%%%%%%%%%%%%%%%%%%%%%%%%%%%%%%%%%%%%%%%%%%%%%%%%%%%%%%%
%Numeracion en romanos
\renewcommand{\thepage}{\roman{page}}
\setcounter{page}{1}
\pagestyle{plain} 

%INDICE 

\tableofcontents

%INDICE DE FIGURAS
\newpage{\pagestyle{empty}}

\listoffigures

% INDICE DE TABLAS
\newpage{\pagestyle{empty}}

\listoftables

%%%%%%%%%%%%%%%%%%%%%%%%%%%%%%%%%%%%%%%%%%%%%%%%%%%%%%%%%%%%%%%%%%%%%%%%%%%%%%%
\newpage{\pagestyle{empty}}

%%%%%%%%%%%%%%%%%%%%%%%%%%%%%%%%%%%%%%%%%%%%%%%%%%%%%%%%%%%%%%%%%%%%%%%%%%%%%%%
\newpage
\thispagestyle{empty}

%Numeracion a partir del capitulo I
\renewcommand{\thepage}{\arabic{page}}
\setcounter{page}{1}
\pagestyle{plain}

\chapter{\LARGE Introducción}
\label{chapter:intro}

\input{cap1.tex}
%%%%%%%%%%%%%%%%%%%%%%%%%%%%%%%%%%%%%%%%%%%%%%%%%%%%%%%%%%%%%%%%%%%%%%%%%%%%%%%
\newpage{\pagestyle{empty}}
\thispagestyle{empty}

\chapter{\LARGE Título del Capítulo 2}
\label{chapter:dos}

\input{cap2.tex}

%%%%%%%%%%%%%%%%%%%%%%%%%%%%%%%%%%%%%%%%%%%%%%%%%%%%%%%%%%%%%%%%%%%%%%%%%%%%%%%
\newpage{\pagestyle{empty}}
\thispagestyle{empty}

\chapter{\LARGE Título del Capítulo 3}
\label{chapter:tres}

\input{cap3.tex}

%%%%%%%%%%%%%%%%%%%%%%%%%%%%%%%%%%%%%%%%%%%%%%%%%%%%%%%%%
\newpage{\pagestyle{empty}}
\thispagestyle{empty}

\chapter{\LARGE Título del Capítulo 4}
\label{chapter:cuatro}

\input{cap4.tex}

%%%%%%%%%%%%%%%%%%%%%%%%%%%%%%%%%%%%%%%%%%%%%%%%%%%%%%%%%
\newpage{\pagestyle{empty}}
\thispagestyle{empty}

\chapter{\LARGE Conclusiones y líneas futuras}
\label{chapter:Resultados}

\input{cap5.tex}

%%%%%%%%%%%%%%%%%%%%%%%%%%%%%%%%%%%%%%%%%%%%%%%%%%%%%%%%%
\newpage{\pagestyle{empty}}
\thispagestyle{empty}

\chapter{\LARGE Summary and Conclusions}
\label{chapter:Conclusiones}

\input{cap6.tex}

%%%%%%%%%%%%%%%%%%%%%%%%%%%%%%%%%%%%%%%%%%%%%%%%%%%%%%%%%
\newpage{\pagestyle{empty}}
\thispagestyle{empty}

\chapter{\LARGE Presupuesto}
\label{chapter:presupuesto}

\input{cap7.tex}

%%%%%%%%%%%%%%%%%%%%%%%%%%%%%%%%%%%%%%%%%%%%%%%%%%%%%%%%%
\newpage{\pagestyle{empty}\cleardoublepage}
\thispagestyle{empty}

\begin{appendix}

\chapter{\LARGE Título del Apéndice 1}
\label{appendix:1}
\input{apendice1.tex}

\chapter{\LARGE Título del Apéndice 2}
\label{appendix:2}
\input{apendice2.tex}

\end{appendix}

%%%%%%%%%%%%%%%%%%%%%%%%%%%%%%%%%%%%%%%%%%%%%%%%%%%%%%%%%%
\begin{thebibliography}{X}
% Aquí figurará la bibliografía
\end{thebibliography}
%%%%%%%%%%%%%%%%%%%%%%%%%%%%%%%%%%%%%%%%%%%%%%%%%%%%%%%%%%

\end{document}

